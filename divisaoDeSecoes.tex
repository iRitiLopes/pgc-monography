\chapter{Introdução}
\label{intro}

% comentário
% 2 ou 3 parágrafos de motivação: ODS e citar até 4 trabalhos relacionados

% 1 parágrafo: 3 etapas do design de sistemas - IHC  - análise da situação atual, síntese da intervenção e avaliação.

% 1 parágrafo: este trabalho está na síntese da intervenção - escrever sobre co-criação e design participativo. Este trabalho teve co-criação

% parágrafo para apresentar conceitos de co-criação e citar +-4 trabalhos relacionados à co-criação

% parágrafo: jogo como proposta de aprendizado sobre reciclagem - citar +- 4 trabalhos - mencionar que a literatura carece de trabalhos sobre reciclagem de lixo eletrônico

% parágrafo: descrever o projeto What Weee Are do Alessio

\section{Justificativa}
\label{justificativa}

%paragrafo: conectar os ODSs, inserir IHC no tema, jogos como abordagem de aprendizado sobre reciclagem e desenvolvimento técnico

\section{Objetivos}
\label{objetivos}

% Objetivo geral é aplicar o co-design para o desenvolvimento de um sistema educativo
%Objetivo específico: teórico - co-criação de IHC; técnico - utilizar Unity como ferramenta de digitalização de ações do projeto What Weee Are

\section{Organização de trabalho}

% Este documento está organizado da seguinte forma: o capitulo~\ref{trabsRelacionados} aborda os... O Capitulo~\ref{metodologia}... O cap~\ref{baseTeorica} (personas, cenários, casos de uso, análise hierárquica de tarefas e descrever as reuniões com o Alessio) O Capítulo~\ref{desenvolvimento} apresenta os conceitos técnicos aobrdados neste trabalho. O Cap~\ref{resultados}... O Cap~\ref{conclusoes}...