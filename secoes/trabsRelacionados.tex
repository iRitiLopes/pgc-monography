\chapter{Trabalhos Relacionados}
\label{trabalhos relacionados}

\par
Jogos eletrônicos têm sido utilizados como ferramenta de ensino e aprendizado em diversos contextos educacionais. Estudos, como o de \cite{gameUseOnSchool}, destacam a eficácia dos jogos cognitivos na promoção da motivação, envolvimento e engajamento dos alunos, bem como no desenvolvimento de habilidades e conhecimentos. O uso de jogos cognitivos para o aprimoramento das habilidades cognitivas em alunos, bem como para o incentivo de comportamentos mais colaborativos e para o aumento da motivação em relação às atividades escolares. A experiência realizada com o uso de tablets revelou-se uma situação de novidade e curiosidade para muitas crianças, as quais se adaptavam facilmente aos menus, à navegação e aos recursos disponíveis. O estudo realizado por José Américo Neto \cite{joseamericonetoGameOds} também aborda a incorporação dos Objetivos de Desenvolvimento Sustentável (ODS) da ONU \cite{odsbrasil} no Brasil por meio do jogo Pokémon GO. No entanto, são discutidos desafios e oportunidades relacionados à gamificação em contextos educacionais, incluindo a necessidade de equilibrar a ludicidade dos jogos com objetivos de aprendizado sérios. Em outro estudo sobre contaminação por resíduos eletrônicos \cite{eWasteContamination}, destaca-se o impacto socioambiental causado pela falta de gestão adequada desses resíduos, bem como a forma como o consumismo acelerado trouxe essa problemática para o âmbito educacional.

Além disso, a pesquisa conduzida por José Américo Neto destaca a aplicação dos Objetivos de Desenvolvimento Sustentável da ONU no contexto brasileiro, por meio do jogo Pokémon GO. No entanto, o estudo também levanta questões relevantes relacionadas aos desafios e oportunidades da gamificação no campo educacional, enfatizando a importância de encontrar um equilíbrio entre a abordagem lúdica dos jogos e os objetivos de aprendizado sérios.

Por outro lado, o estudo sobre contaminação por resíduos eletrônicos ressalta o impacto socioambiental causado pela falta de gerenciamento adequado desses materiais, destacando como o aumento do consumismo tem levado essa problemática para o cenário educacional. Essas investigações revelam a necessidade de conscientização e educação sobre a gestão adequada do lixo eletrônico, bem como a importância de abordar essa questão por meio de estratégias educacionais eficazes e inovadoras.

O trabalho de \cite{savi2008jogos} propõe que a existência de mídias modernas pode seduzir à participação e absorção dos conteúdos de forma lúdica. Já \cite{Vasconcellos_Carvalho_Barreto_Atella_2017} propõe que a interação proporcionada pelos jogos tem um papel crucial na aquisição de conhecimento, trazendo um paralelo com o mundo real, onde o jogo propõe suas regras explícitas e implícitas para que seja possível a sobrevivência no mesmo, de forma similar com as regras do mundo real, que podem ser alteradas, refinadas e criadas. Assim, o jogador absorve regras não verbalizadas, que podem ser usadas como fonte de inspiração para contribuir com as regras do mundo real. A utilização de jogos com temas como a reciclagem, exemplificada em \cite{Raio_2016}, pode aumentar o engajamento dos alunos e torná-los mais ativos em jogos educacionais, sendo uma maneira de inclusão também para alunos surdos \cite{silva2013uso}.

Os jogos também têm sido utilizados como ferramenta para a educação ambiental, utilizando de ferramentas e práticas de co-criação para atingir os objetivos, como mostrado em \cite{rock2018multidisciplinary} e \cite{PRAHALAD20045}. A co-criação e o envolvimento de participantes de diversas áreas incentivam a disrupção de ideias, produzindo produtos de melhor qualidade para o receptor final, além de criar um melhor relacionamento entre as partes envolvidas no processo co-criativo.

A reciclagem de materiais eletrônicos também é um tema importante na atualidade devido ao aumento da quantidade de resíduos eletrônicos gerados e às preocupações ambientais e de saúde associadas ao descarte inadequado desses resíduos. O trabalho de \cite{eWasteContamination} examina as práticas atuais de gerenciamento de resíduos eletrônicos e propõe soluções potenciais para melhorar a gestão desses resíduos, discutindo os desafios associadas ao descarte inadequado desses materiais. É necessário encontrar soluções para melhorar a gestão desses resíduos e conscientizar a população sobre a importância da reciclagem de eletrônicos. A co-criação entre diferentes áreas também é importante para a produção de jogos educacionais e ambientais de qualidade, além de incentivar a disrupção de ideias. Portanto, é fundamental explorar o potencial dos jogos eletrônicos como uma ferramenta educacional e ambiental, levando em consideração os desafios e oportunidades associados à gamificação em contextos educacionais e a necessidade de equilibrar o lado lúdico dos jogos com objetivos de aprendizado sérios.

Existem diversos trabalhos relacionados a jogos eletrônicos educativos com o tema de reciclagem. \cite{santos2012} analisou sete jogos eletrônicos voltados à temática da educação ambiental e concluiu que os jogos apresentados podem ser utilizados como ferramentas pedagógicas para a conscientização sobre a importância da preservação do meio ambiente \cite{santos2012}. Além disso, jogos educativos podem associar a função lúdica à pedagógica no ensino de geografia, especialmente na educação ambiental, para assim constituir-se em um recurso motivador da aprendizagem, complementando o saber, o conhecimento e a descoberta do mundo pela criança.

Além disso, \cite{souza2016} apresentam a educação ambiental como ferramenta para o manejo de resíduos sólidos no cotidiano escolar \cite{souza2016}. O estudo de \cite{goletando2018} apresenta um jogo educacional para o ensino da coleta seletiva e reciclagem de resíduos sólidos, que contribui para o aprendizado dos alunos na área de educação ambiental \cite{goletando2018}.
Portanto, há diversos trabalhos que destacam a importância dos jogos eletrônicos educativos para a conscientização sobre a importância da reciclagem de lixo e a preservação do meio ambiente.

Existem diversos trabalhos acadêmicos que abordam a eficácia do uso de jogos eletrônicos na educação. Um desses trabalhos é "A importância dos jogos digitais na educação", que visa demonstrar como os jogos digitais podem auxiliar no aprendizado, desde a alfabetização até o ensino superior \cite{importancia-jogos-digitais}. O trabalho aborda os principais desafios para o desenvolvimento dos jogos educacionais, como a gamificação da aprendizagem e a interatividade, além de apresentar exemplos de jogos educacionais \cite{importancia-jogos-digitais}.

Em "Uso de jogos educacionais no processo de ensino e de aprendizagem", que destaca a importância da utilização de jogos digitais, principalmente os educacionais, na educação, proporcionando ao aluno motivação e desenvolvendo hábitos de persistência no aprendizado \cite{uso-jogos-educacionais}. O trabalho também aborda a necessidade de uma organização prévia para trabalhar com jogos educacionais, definindo objetivos e a estratégia pedagógica \cite{uso-jogos-educacionais}. Ao mesmo tempo que "O uso dos jogos para a aprendizagem no ensino superior", que investiga como a utilização de metodologias ativas/jogos está sendo utilizada por jovens universitários e seus professores \cite{uso-jogos-aprendizagem}. O estudo destaca a importância das instituições de ensino estarem atentas às novas exigências sociais e tecnológicas, buscando mudanças nas estratégias metodológicas para a educação \cite{uso-jogos-aprendizagem}. O trabalho conclui que os jogos educativos são uma excelente ferramenta educacional, proporcionando benefícios como socialização, cooperação, criatividade, interatividade e interdisciplinaridade \cite{uso-jogos-aprendizagem}.

Além desses trabalhos, há também um estudo sobre a utilização de jogos eletrônicos nas aulas de educação física \cite{jogos-educacao-fisica}. O estudo destaca a falta de interesse dos professores em utilizar jogos eletrônicos durante as aulas de educação física \cite{jogos-educacao-fisica}. O trabalho conclui que a maior dificuldade para trabalhar a temática jogo eletrônico nas aulas de educação física é a falta de equipamentos adequados para tal uso, sendo eles tanto a falta de materiais disponíveis pela escola quanto a falta de formação dos professores \cite{jogos-educacao-fisica}.

Os jogos eletrônicos têm sido amplamente utilizados como ferramentas de ensino e aprendizado em diversos contextos educacionais. Diversos estudos destacam a eficácia dos jogos cognitivos na promoção da motivação, envolvimento e engajamento dos alunos, bem como no desenvolvimento de habilidades e conhecimentos. O uso de jogos cognitivos é uma excelente estratégia para o aprimoramento das habilidades cognitivas em alunos, bem como para o incentivo de comportamentos mais colaborativos e para o aumento da motivação em relação às atividades escolares. Alguns trabalhos propõem que a interação proporcionada pelos jogos tem um papel crucial na aquisição de conhecimento, trazendo um paralelo com o mundo real, onde o jogo propõe suas regras explícitas e implícitas para que seja possível a sobrevivência no mesmo, de forma similar com as regras do mundo real. A utilização de jogos com temas como a reciclagem pode aumentar o engajamento dos alunos e torná-los mais ativos em jogos educacionais, sendo uma maneira de inclusão também para alunos com necessidades especiais. Além disso, os jogos também têm sido utilizados como ferramenta para a educação ambiental, utilizando ferramentas e práticas de co-criação para atingir os objetivos. A reciclagem de materiais eletrônicos também é um tema importante na atualidade devido ao aumento da quantidade de resíduos eletrônicos gerados e às preocupações ambientais e de saúde associadas ao descarte inadequado desses resíduos. É fundamental explorar o potencial dos jogos eletrônicos como uma ferramenta educacional e ambiental, levando em consideração os desafios e oportunidades associados à gamificação em contextos educacionais e a necessidade de equilibrar o lado lúdico dos jogos com objetivos de aprendizado sérios.
