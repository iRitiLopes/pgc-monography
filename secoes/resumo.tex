
% ---
% RESUMOS
% ---

% RESUMO em português
\setlength{\absparsep}{18pt} % ajusta o espaçamento dos parágrafos do resumo
\begin{resumo}

O trabalho apresenta o desenvolvimento de um jogo eletrônico educativo voltado para a educação em reciclagem de resíduos eletrônicos. O objetivo principal é abordar a problemática do descarte inadequado do lixo eletrônico e promover a conscientização sobre a importância da reciclagem desses materiais. O jogo foi desenvolvido utilizando o motor gráfico Unity e a linguagem de programação C\#.

O estudo aborda a eficácia dos jogos cognitivos na promoção da motivação, envolvimento e engajamento dos alunos, bem como no desenvolvimento de habilidades e conhecimentos. Além disso, explora a interação proporcionada pelos jogos e seu papel crucial na aquisição de conhecimento. O trabalho também destaca o impacto socioambiental causado pela falta de gerenciamento de lixo eletrônico e as oportunidades e desafios associados à gamificação em contextos educacionais.

Como resultado, o jogo "What Weee Are" foi apresentado a estudantes do ensino fundamental e médio, gerando um maior interesse e envolvimento dos alunos no tema da reciclagem de materiais eletrônicos. A análise hierárquica de tarefas foi utilizada para desenvolver as mecânicas e a progressão do jogo.

 \textbf{Palavras-chaves}: lixo eletrônico, jogo, jogo educativo, computação, ecologia, resíduo eletrônico, REEE, e-lixo.
\end{resumo}

% ABSTRACT in english
\begin{resumo}[Abstract]
 \begin{otherlanguage*}{english}
The paper presents the development of an educational electronic game focused on electronic waste recycling education. The main objective is to address the problem of inadequate disposal of electronic waste and promote awareness of the importance of recycling these materials. The game was developed using the Unity graphics engine and the C\# programming language.

The study addresses the effectiveness of cognitive games in promoting student motivation, engagement, and involvement, as well as in the development of skills and knowledge. Additionally, it explores the interaction provided by games and their crucial role in knowledge acquisition. The paper also highlights the socio-environmental impact caused by the lack of electronic waste management and the opportunities and challenges associated with gamification in educational contexts.

As a result, the game "What Weee Are" was presented to elementary and high school students, generating greater interest and engagement among students in the topic of electronic materials recycling. The hierarchical task analysis was used to develop the mechanics and progression of the game.
   \vspace{\onelineskip}
 
   \noindent 
   \textbf{Keywords}: electronic waste, game, educational game, computing, ecology, electronic waste, WEEE, e-waste.
 \end{otherlanguage*}
\end{resumo}