\chapter{Considerações Finais}
\label{conclusao}

Neste capítulo, são apresentadas as possibilidades existentes que ainda podem ser exploradas no desenvolvimento do jogo em futuras versões, além de disponibilizar o código-fonte do projeto para que todos possam contribuir.

What Weee Are é um jogo com infinitas possibilidades. A adição de cenários, itens e novos aprimoramentos ao personagem são apenas algumas das opções. O Unity Analytics pode ser uma poderosa ferramenta capaz de propor a implementação de testes A/B durante o jogo, coletar dados a partir de cada teste e realizar estudos com base nas decisões dos jogadores para continuar refinando e melhorando o jogo no futuro.

\section{Trabalhos Futuros}
Em What Weee Are, a expansão do universo é um fato. Os maiores desenvolvimentos neste estágio atual foram a implementação de mecânicas básicas e estruturas que proporcionam justamente a facilidade ao adicionar novos recursos no jogo. A adição de novos itens e receitas basta adicionar novos itens nos arquivos JSON que definem os itens, e a adição de novos diálogos se torna a adição de um novo GameObject do tipo Dialogue. Porém, no futuro, há a possibilidade de levá-lo para dispositivos móveis e também a captação de Analytics (coletar dados de tomada de decisão, tempo de jogo, coleta dos itens, derrota dos inimigos) para servir de insumo ao construir novas fases e também à evolução de aprendizado dos jogadores, conforme o progresso do jogo, por meio da ferramenta robusta oferecida pela própria UNITY, o UNITY ANALYTICS \footnote{Veja mais em: https://unity.com/products/unity-analytics}.

O uso dos dados gerados a partir do \textit{analytics} poderá proporcionar a implementação de testes A/B durante o jogo, para, enfim, analisar a implementação e o resultado de novas abordagens, implementação de modos mais desafiadores e recompensadores aos jogadores. Isso poderá trazer análises sobre as decisões tomadas em pontos-chave do jogo e, assim, projetar melhorias aos cenários e, por fim, proporcionar uma melhor experiência aos jogadores.

What Weee Are é um jogo de código aberto, pronto para receber novas ideias e implementações de desenvolvedores e jogadores que se sintam à vontade para contribuir com a progressão e novos estágios do jogo. O projeto pode ser encontrado aqui: \url{https://github.com/iRitiLopes/WhatWeeAre}


\section{Possibilidades}
\begin{itemize}
    \item Desenvolvimento e teste do jogo educativo "What Weee Are" para avaliar sua eficácia em fornecer informações e conscientização sobre a reciclagem de resíduos eletrônicos e suas consequências para o meio ambiente.
    \item Análise da eficácia dos jogos educativos no contexto dos Objetivos de Desenvolvimento Sustentável da ONU em comparação com outros métodos de conscientização, como palestras e vídeos informativos.
    \item Estudo da importância da cocriação no desenvolvimento de jogos educativos e digitais, e sua influência na aceitação e eficácia desses jogos.
    \item Investigação do uso de ferramentas de análise de dados, como o Unity Analytics, na avaliação do desempenho do jogador e na coleta de feedback do usuário, para melhorar a eficácia do jogo educativo e a experiência do usuário.
    \item Desenvolvimento de uma plataforma de jogos educativos para a conscientização ambiental e sustentabilidade, permitindo que outros desenvolvedores contribuam com jogos e conteúdos educacionais.
    \item Avaliação do impacto social e ambiental dos jogos educativos na conscientização e mudança de comportamento dos jogadores em relação à sustentabilidade e reciclagem de resíduos eletrônicos.
    \item Exploração de outras tecnologias, como realidade virtual ou aumentada, para aprimorar a experiência do jogador e aumentar o impacto dos jogos educativos no público-alvo.
    \item Investigação de estratégias para incentivar o uso de jogos educativos em escolas e programas educacionais formais para apoiar a conscientização e educação em sustentabilidade e desenvolvimento sustentável.
    \item Comparação da eficácia dos jogos educativos em diferentes contextos, como faixa etária, nível educacional, contexto cultural e socioeconômico, e como essa informação pode ser usada para aprimorar o design e a disseminação dos jogos educativos.
\end{itemize}