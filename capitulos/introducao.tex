% ----------------------------------------------------------
% Introdução 
% Capítulo sem numeração, mas presente no Sumário
% ----------------------------------------------------------

\chapter*[Justificativa]{Justificativa}
\addcontentsline{toc}{chapter}{Justificativa}

Além do desperdício de resíduos gerados pela sociedade humana é um problema a ser enfrentado, desde a quarta revolução[1] e o massivo número de produtos eletrônicos sendo produzidos e descartados, passou a ser um problema o que fazer com esses resíduos novos, pilhas, baterias, celulares, televisores e computadores por exemplo. Existe a preocupação maior com estes descartados, pois em sua vez, possuem componentes internos como baterias e outros elementos que podem ser extremamente danosos ao meio ambiente, além de possuir metais preciosos e outros elementos raros e que poderiam ser reaproveitados na manufatura de outros componentes.

Mas como mudar esse problema? Podemos incentivar as pessoas por meio da sensibilização e conscientização por meio da educação[2] mostrou um ganho na compreensão e mudanças de hábitos. Existe uma proposta de jogo eletrônico baseado em Novas Tecnologias da Informação e Comunicação, como uma ferramenta para auxiliar professores para abordar este tipo de assunto de uma forma mais interativa com os seus alunos.[3] Seguindo este exemplo: What Weee Are, é um jogo eletrônico que aborda o assunto dos resíduos eletrônicos e pode ser usado como uma ferramenta na conscientização das pessoas quanto ao assunto de resíduos eletrônicos.


\section*{Motivação}\label{sec:motivacao}
\addcontentsline{toc}{section}{Motivação}

O motivo de utilizar um jogo como motor para impulsionar o interesse dos mais jovens para questões ambientais como essa, parte de que os jogos estão amplamente presentes na vida das pessoas, e podem desenvolver habilidades de cooperação, colaboração em desafios para resolver problemas[4]. Podendo ser uma ferramenta utilizada pelos professores para fomentar o desenvolvimento cognitivo dos alunos, pois mostra uma ferramenta capaz de motivar e despertar o interesse à diversos assuntos.[5]

\section*{Objetivos}\label{sec:objetivos}
\addcontentsline{toc}{section}{Objetivos}

\lipsum[36]
