% ---
\chapter{WhatWeeeAre: jogo eletrônico com abordagem à reciclagem de resíduos eletrônicos}
% ---

What Weee Are é um projeto multimídia, um jogo eletrônico que aborda os desperdício de recursos de nossa sociedade, a ideia é que o jogo seja capaz de ser usado como uma ferramenta educacional utilizada pelos professores, capaz de motivar o interesse dos alunos a um dos problemas atuais de nossa sociedade. Utilizar um jogo como ferramenta de educação já foi apresentado como viável e devemos aproveitar as oportunidades que o atual mundo tecnológico nos apresenta[5]. 

What Weee Are é um jogo de plataforma 2D, como elementos de RPG para progressão do personagem, sendo capaz de coletar recursos chaves(ouro, cobre, ferro e plantas) que podem ser utilizados para comprar aprimoramentos ao personagem. Outra maneira de conseguir recursos é através da coleta de materiais não brutos como: cabos, placas, mouse. E desmontá-los a fim de obter seus recursos brutos. Utilizando fotografias reais como material gráfico para compor as cenas dos jogos, personagem, recursos, obstáculos.

%-
\section{Metodologias}
%-

Para o desenvolvimento do projeto, foi utilizado o motor gráfico UNITY, disponibilizado gratuitamente em https://unity.com/pt, com auxilio da linguagem de programação C\# para criação de scripts e utilitários ao jogo.

\subsection{Características necessárias}
Durante a concepção do jogo, foram analisados os conceitos de jogos de plataformas que fizeram sucesso e tomadas como inspiração para o desenvolvimento, como a franquia Donkey Kong Country e Super Mario Bros, ambas pertencentes à Nintendo e jogos de RPG onde existem uma progressão do personagem e novas habilidades adquiridas, como a série The Legend of Zelda, Final Fantasy:

\begin{itemize}
    \item Tipos diferentes de pulo, como por exemplo segurar o botão de pular e efetuar um salto longo.
    \item Coleta de itens e sistema de Crafting
    \item Habilidades desbloqueáveis
    \item Plataformas dinâmicas
    \item Inimigos de diferentes aspectos
    \item Movimentação realizada pelo jogador
\end{itemize}

\subsection{Objetivos do jogo}
O Objetivo de What Weee Are é o jogador passar pelas 4 fases do jogo. Em cada fase o jogador será confrontado com um número de inimigos e itens a serem coletados. Nas fases estarão a disposição as salas de Assembler e Disassembler para auxiliá-lo na obtenção de itens complexos e novas habilidades. Com auxílio do Grilo Falante, o mentor do personagem, será apresentada a narrativa do jogo e explicando os eventos do universo a cada nova fase encontrada. 

Em cada fase, os desafios serão incrementados e necessitará que o jogador consiga as habilidades especiais disponíveis pelo sistema de Crafting, para acessar pontos em que sem tais habilidades se tornará impossível, forçando com que o jogador interaja com o ambiente e colete o maior número de itens possíveis nas fases. 

Há dois tipos de inimigos nas fases, as Formigas, e as Aranhas, e suas variantes, que possuem características diferentes e dificuldades incrementadas a cada nova fase. Quando o jogador morre (atinge o limite mínimo de vidas) ele retorna ao início da fase.


\subsection{Desenvolvimento}
Durante todo o processo de desenvolvimento, foram efetuados testes e questionamentos sobre conceitos que são necessários para incrementar a experiência do jogador.

Inicialmente o projeto foi iniciado desenvolvendo mecânicas que poderiam ser reutilizadas ao longo de todo o progresso do jogo, mecânicas essas que foram sendo aprimoradas durante o desenvolvimento e conforme era necessário alguma nova característica se tornasse necessária para progredir. 

Inicialmente foram desenvolvidas as interações do personagem Weee com o ambiente e os inimigos, interações essas de morte ao chegar à 0(zero) vidas disponíveis, dano ao chocar-se com inimigos pelas laterais e por cima, e podendo derrotar os inimigos ao pular sobre eles. Nessas interações com os inimigos, ao tomar dano o inimigo será levemente empurrado na direção em que fora atingido pelo Weee, o mesmo se aplica ao Weee ao ser atingido pelos inimigos, essa mecânica se chama knockback.

Assim fora iniciado o desenvolvimento do sistema de coleta de itens e de inventário do personagem, onde seria possível visualizar a quantidade de diferentes itens coletados. 

<imagem 1>

Os itens coletados podem ser refinados em itens mais complexos ou desmembrados em itens mais primitivos, esta definição pode ser efetuada a partir de como o item é produzido, este sistema de receitas em conjunto aos sistemas de montagem e desmontagem, que irão consumir os itens produtores para a criação dos itens resultantes:

<imagem 2>\\
<imagem 3>\\
<imagem 4>\\

Para tudo isso poder funcionar e ter a dinâmica de levar os itens aos compartimentos para montá-los e desmontá-los, foi utilizado o conceito de DragAndDrop onde o usuário poderá clicar e arrastar os itens até os compartimentos de entrada.

% ---
\subsection{Tecnologias utilizadas}
Foi utilizado a engine para jogos Unity, versão 2021, focado nas ferramentas em 2D, como:
\begin{itemize}
    \item Tile Pallet: ferramenta de paleta para aplicar os Tiles (imagens/sprites) no jogo 2D, dentro do cenário, utilizando um pincel é permitido pintar obstáculos e plataformas de maneira facilitada ao desenvolvimento.
    \item Tile rule: ferramenta auxiliar ao tile pallet, que ajuda a criar uma regra de quais sprites devem ficar seguindo um ao outro e montar um cenário coerente com bordas, segmentos, rampas, etc.
    \item Animators: Ferramenta auxiliar para criar animações dentro do jogo, como animação de andar, pular, morrer.
    \item Localization: Ferramenta disponibilizada pela Unity, que auxilia na localização do jogo em diversos idioma, possibilitando assim o desenvolvedor a promover a localização do jogo em diversas linguas com mais facilidades, por meio de tabelas de texto e também tabelas de objetos, podendo conter texturas por exemplo.
\end{itemize}

\subsection{Implementação}
\lipsum[34-48]




